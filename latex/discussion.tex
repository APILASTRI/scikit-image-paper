%-----------
% Discussion
%-----------

\section*{Discussion}
  \label{sec:discussion}

  \subsection*{Related work}

  In this section, we describe other libraries with similar goals to ours.
  
  Within the scientific Python ecosystem, Mahotas contains many similar
  functions, and is furthermore also designed to work with NumPy arrays
  \citep{Mahotas}. The major philosophical difference between Mahotas and
  scikit-image is that Mahotas is almost exclusively written in templated C++,
  while scikit-image is written in Python and Cython. We feel that our choice
  lowers the barrier of entry for new contributors. However,
  thanks to the interoperability between the two provided by the NumPy array
  data format, users don't have to choose between them, and can simply use the
  best components of each.

  ImageJ and its batteries-included Fiji distribution are probably the most
  popular open-source tools for image analysis \citep{imagej,Fiji}. Although 
  Fiji's breadth of functionality is unparalleled, it is centered around 
  interactive, GUI use. For many developers, then, scikit-image offers several
  advantages. Although Fiji offers a
  programmable macro mode that supports many scripting languages, many of the
  macro functions activate GUI elements and cannot run in headless mode. This
  is problematic for data analysis in high-performance computing
  cluster environments or web backends, for example. Additionally, Fiji's
  inclusive plugin policy results in an inconsistent API and variable
  documentation quality. For example,
  some plugins take image file paths as input, while others work on the
  currently active image. Using scikit-image to develop new functionality or
  to build batch applications for distributed computing is often much simpler,
  thanks to its consistent API and the wide distribution of the scientific
  Python stack.

  In many respects, the image analysis toolbox of the Matlab environment is
  quite similar to scikit-image. For example, its API is mostly functional and
  applies to generic multidimensional numeric arrays. However, Matlab's
  commercial licensing can be a significant nuisance to users. Additionally,
  the licensing cost increases dramatically for parallel computing, with
  per-worker pricing\footnote{\url{http://www.mathworks.com.au/products/distriben/description3.html}, Accessed 2014-05-09}.
  Finally, the closed source nature of the toolbox prevents users from
  learning from the code or modifying it for specific purposes, which is a
  common necessity in scientific research. We refer readers back to the
  Development Practices section for a summary of the practical and
  philosophical advantages of our open-source licensing.

  OpenCV is an open-source computer vision library with a separate image
  processing module, free for use under the BSD-license \citep{opencv}. It is
  developed in C/C++ and the project's aim is to provide implementations for
  real-time applications, which makes OpenCV a library with comparatively high
  barrier of entry for code study and modification. The library provides
  interfaces for several high-level programming languages, including Python
  through the NumPy-array data-type for images. The Python interface is
  essentially a one-to-one copy of the underlying C/C++ API, and thus image
  processing pipelines have to follow an imperative programming style. As
  opposed to scikit-image, which provides a Pythonic interface with the option
  to follow an imperative or functional approach. Beyond that, OpenCV's image
  processing module is traditionally limited to 2-dimensional imagery.

  The choice of image processing package depends on several factors, including
  speed, code quality and correctness, community support, ecosystem, feature
  richness, and users' ability to contribute. Sometimes, advantages in one
  factor come at the cost of another. For example, our approach of writing code
  in a high-level language may affect performance, or our strict code review
  guidelines may hamper the number of features we ultimately provide. We
  motivate our design decisions for scikit-image in the Development Practices
  section, and leave readers to decide which library is right for them.
