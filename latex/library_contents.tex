%-----------------
% Library Contents
%-----------------

\section*{Library contents}
  \label{library-contents}

  The scikit-image project started in August of 2009 and has received contributions from more than 100 individuals \citep{ohloh}.  The package can be installed from, amongst other sources, the Python Package Index, Continuum Anaconda \citep{anaconda}, Enthought Canopy \citep{canopy}, Python(x,y) \citep{pythonxy}, NeuroDebian \citep{neurodebian} and GNU/Linux distributions such as Ubuntu \citep{ubuntu}. In March 2014 alone, the package was downloaded more than 5000 times from the Python Package Index \citep{pypi}.

  The package currently contains the following sub-modules:

  \begin{itemize}

    \item color: Color space conversion.
    \item data: Test images and example data.
    \item draw: Drawing primitives (lines, text, etc.) that operate on NumPy arrays.
    \item exposure: Image intensity adjustment, e.g., histogram equalization, etc.
    \item feature: Feature detection and extraction, e.g., texture analysis, corners, etc.
    \item filter: Sharpening, edge finding, rank filters, thresholding, etc.
    \item graph: Graph-theoretic operations, e.g., shortest paths.
    \item io: Reading, saving, and displaying images and video.
    \item measure: Measurement of image properties, e.g., similarity and contours.
    \item morphology: Morphological operations, e.g., opening or skeletonization.
    \item novice: Simplified interface for teaching purposes.
    \item restoration: Restoration algorithms, e.g., deconvolution algorithms, denoising, etc.
    \item segmentation: Partitioning an image into multiple regions.
    \item transform: Geometric and other transforms, e.g., rotation or the Radon transform.
    \item viewer: A simple graphical user interface for visualizing results and exploring parameters.

  \end{itemize}
