%--------------------------
% Usage Examples: Education
%--------------------------

  \subsection*{Education}
    \label{education}

    scikit-image's simple, well-documented application programming interface (API) makes it ideal for educational use, via self-taught exploration or formal training sessions.

    The online gallery of examples not only provides an overview of the functionality available in the package but also introduces many of the algorithms commonly used in image processing. This visual index also helps beginners overcome a common entry barrier: locating the class (denoising, segmentation, etc.) and name of operation desired, without being proficient with image processing jargon.  For many functions, the documentation includes links to research papers or Wikipedia pages to further guide the user.

    Demonstrating the broad utility of scikit-image in education, thirteen-year-old Rishab Gargeya of the Harker School won the Synopsys Silicon Valley Science and Technology Championship using scikit-image in his project, ``A software based approach for automated pathology diagnosis of diabetic retinopathy in the human retina'' \citep{sciencefair}.

    We have also delivered image processing tutorials using scikit-image at various annual scientific Python conferences, such as PyData 2012, SciPy India 2012, and EuroSciPy 2013. Course materials for some of these sessions are found in \cite{scipylecturenotes} and are licensed under the permissive CC-BY license \citep{cc-by}. These typically include an introduction to the package and provide intuitive, hands-on introductions to image processing concepts. The well documented application programming interface (API) along with tools that facilitate visualization contribute to the learning experience, and make it easy to investigate the effect of different algorithms and parameters. For example, when investigating denoising, it is easy to observe the difference between applying a median filter (\texttt{filter.rank.median}) and a Gaussian filter (\texttt{filter.gaussian\_filter}), demonstrating that a median filter preserves straight lines much better.

    Finally, easy access to readable source code gives users an opportunity to learn how algorithms are implemented and gives further insight into some of the intricacies of a fast Python implementation, such as indexing tricks and look-up tables.
