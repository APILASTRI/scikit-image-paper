%-------------
% Introduction
%-------------

\section*{Introduction}
  \label{sec:introduction}

  In our data-rich world, images represent a significant subset of all measurements made. Examples include DNA microarrays, microscopy slides, astronomical observations, satellite maps, robotic vision capture, synthetic aperture radar images, and higher-dimensional images such as 3-D magnetic resonance or computed tomography imaging. Exploring these rich data sources requires sophisticated software tools that should be easy to use, free of charge and restrictions, and able to address all the challenges posed by such a diverse field of analysis.

  This paper describes scikit-image, a collection of image processing algorithms implemented in the Python programming language by an active community of volunteers and available under the liberal BSD Open Source license. The rising popularity of Python as a scientific programming language, together with the increasing availability of a large eco-system of complementary tools, make it an ideal environment in which to produce an image processing toolkit.

  The project aims are:

  \begin{enumerate}
    \item  % First item
      \textit{To provide high quality, well-documented and easy-to-use implementations of common image processing algorithms.}

      Such algorithms are essential building blocks in many areas of scientific research, algorithmic comparisons and data exploration. In the context of reproducible science, it is important to be able to inspect any source code used for algorithmic flaws or mistakes. Additionally, scientific research often requires custom modification of standard algorithms, further emphasizing the importance of open source.

    \item  % Second item
      \textit{To facilitate education in image processing.}

      The library allows students in image processing to learn algorithms in a hands-on fashion by adjusting parameters and modifying code. In addition, a \texttt{novice} module is provided, not only for teaching programming in the ``turtle graphics'' paradigm, but also to familiarize users with image concepts such as color and dimensionality. Furthermore, the project takes part in the yearly Google Summer of Code program \citep{gsoc}, where students learn about image processing and software engineering through contributing to the project.

    \item  % Third item
      \textit{To address industry challenges.}

      High quality reference implementations of trusted algorithms provide industry with a reliable way of attacking problems, without having to expend significant energy in re-implementing algorithms already available in commercial packages.  Companies may use the library entirely free of charge, and have the option of contributing changes back, should they so wish.
  \end{enumerate}