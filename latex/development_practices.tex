%----------------------
% Development practices
%----------------------

\section*{Development practices}
  \label{sec:development-practices}

  The purpose of scikit-image is to provide a high-quality library of powerful, diverse image processing tools free of charge and restrictions. These principles are the foundation for the development practices in the scikit-image community.

  The library is licensed under the \emph{Modified BSD license}, which allows unrestricted redistribution for any purpose as long as copyright notices and disclaimers of warranty are maintained \citep{BSD}. It is compatible with GPL licenses, so users of scikit-image can choose to make their code available under the GPL. However, unlike the GPL, it does not require users to open-source derivative work (BSD is not a so-called copyleft license). Thus, scikit-image can also be used in closed-source, commercial environments.

  The development team of scikit-image is an open community that collaborates on the \emph{GitHub} platform for issue tracking, code review, and release management\footnote{\url{https://github.com/scikit-image}}. \emph{Google Groups} is used as a public discussion forum for user support, community development, and announcements\footnote{\url{https://groups.google.com/forum/?&fromgroups\#!forum/scikit-image}}.

  scikit-image complies with the PEP8 coding style standard \citep{PEP8} and the NumPy documentation format \citep{NumpyDoc} in order to provide a consistent, familiar user experience across the library similar to other scientific Python packages. As mentioned earlier, the data representation used is \emph{n}-dimensional NumPy arrays, which ensures broad interoperability within the scientific Python ecosystem. The majority of the scikit-image API is intentionally designed as a functional interface which allows one to simply apply one function to the output of another. This modular approach also lowers the barrier of entry for new contributors, since one only needs to master a small part of the entire library in order to make an addition.

  We ensure high code quality by a thorough review process using the pull
  request interface on GitHub\footnote{\url{https://help.github.com/articles/using-pull-requests}, Accessed 2014-05-15.}.
  This enables the core developers and other interested parties to comment on
  specific lines of proposed code changes, and for the proponents of the
  changes to update their submission accordingly. Once all the changes have
  been approved, they can be merged automatically. This process applies not
  just to outside contributions, but also to the core developers.

  The source code is mainly written in Python, although certain performance critical sections are implemented in Cython, an optimising static compiler for Python \citep{Cython}. scikit-image aims to achieve full unit test coverage, which is above 85\% as of release 0.10 and continues to rise. A continuous integration system\footnote{\url{https://travis-ci.org}, \url{https://coveralls.io}, Accessed 2014-03-30} automatically checks each commit for unit test coverage and failures on both Python 2 and Python 3. Additionally, the code is analyzed by flake8 \citep{flake8} to ensure compliance with the PEP8 coding style standards \citep{PEP8}. Finally, the properties of each public function are documented thoroughly in an API reference guide, embedded as Python docstrings and accessible through the official project homepage or an interactive Python console. Short usage examples are typically included inside the docstrings, and new features are accompanied by longer, self-contained example scripts added to the narrative documentation and compiled to a gallery on the project website. We use Sphinx \citep{Sphinx} to automatically generate both library documentation and the website.

  The development master branch is fully functional at all times and can be obtained from GitHub\footnote{\url{https://github.com/scikit-image/scikit-image}}. The community releases major updates as stable versions approximately every six months. Major releases include new features, while minor releases typically contain only bug fixes. Going forward, users will be notified about API-breaking changes through deprecation warnings for two full major releases before the changes are applied.
